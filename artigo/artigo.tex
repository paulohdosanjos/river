\documentclass{article}
\usepackage{mathtools} %box inside align
\usepackage[sc]{mathpazo}
\usepackage{times}

\input{preamble}

\setlength{\parskip}{1ex} % espaçamento vertical entre parágrafos
\setlength{\parindent}{0pt} % recuo horizontal de parágrafos

\title{\textbf{Explicações do agente em um processo de decisão sequencial}}

\begin{document}

\maketitle

\section{Introdução}

A área de planejamento estocástico considera o problema de decisões sequenciais em ambientes probabilísticos. O resultado de um agente inteligente é uma política, que dependendo da complexidade do espaço de estados, não é interpretável por um humano leigo.

Nesse artigo, buscamos estudar o processo geral de explicação do agente inteligente. Ou seja, as formas nas quais o agente pode explicar ao usuário do sistema sua decisão. Contribuimos com algoritmos que facilitam a interpretação de políticas e a argumentação de escolhas.
Usamos o problema abstrato do rio como base para nossas experimentações.

\section{Background teórico}

Consideramos que o problema do agente inteligente pode ser modelado como um \textbf{processo markoviano de decisão} (MDP). Em um processo desse tipo, há três elementos: estados, ações e custos. O agente começa em um estado inicial e toma alguma ação com base nesse estado, em seguida, o ambiente transita para outro estado de acordo com uma distribuição específica, a qual o agente não tem controle, e o agente recebe um custo numérico. O processo acaba somente se o ambiente transitar para estados especiais, chamados estados meta, que como o nome diz, sinaliza a conclusão do objetivo do agente. 

Agora, definimos formalmente um MDP.

Um \text{MDP} é definido formalmente como uma tupla $M = (S, A, T, G, C)$. Onde $S$ é o conjunto de estados, $A$ é conjunto de ações possíveis, $T$ é a matriz de transição que determina as propabilidades de transição do processo, $G$ é o conjunto de estados objetivo e $C$ é a função custo. 

Então, a função custo $C: S \times A \rightarrow \Re$ determina os custos das transições: $C(s,a)$ fornece o custo quando o agente toma a ação $a$ no estado $s$. A matriz de transições $T: S \times A \times S \rightarrow \Re$ determina a distribuição citada acima: $T(s',a,s')$ é a probabilidade do ambiente transitar para o estado $s'$ quando a ação $a$ é tomada no estado $s$.

Além disso, introduzimos algumas notações e definições adicionais. 

O processo começa no estado $s_0$, então o agente toma a ação $a_0$ de custo $C(s_0,a_0) = c_0$. O processo então transita para um estado $s_1$. Em seguida, o agente toma a ação $a_1$ de custo $c_1$. A sequência de todos esses elementos durante um processo é chamada de \textbf{história}.

\subsection{Definições e notação}

\textbf{Definição 1}. Uma história $h$ de um MDP $M = (S, A , T, G, C)$ é uma sequência
\[h = (s_0,a_0,c_0,s_1,a_1,c_1,\dots)\]

onde $s_i$ é o estado do processo no tempo $i$, $a_i$ é a ação tomada nesse tempo e $c_i$ é o custo obtido, com $i \in \{0,1,2,...\}$. Ou seja, uma história é uma possível ocorrência de processo de $M$.

Para nos referirmos a estados, ações e custos dentro de uma história, definimos a seguir duas notações auxiliares.

\textbf{Definição 2}. Seja $h$ a história de um MDP qualquer. Definimos $s(h,t)$ como o estado em que o processo se encontra no instante $t$. Analogamente, $a(h,t)$ é ação tomada no instante $t$ e $c(h,t)$ o custo obtido. 

\textbf{Definição 3}. Seja $h$ a história de um MDP qualquer. Definimos $f(h,A)$ como a quantidade de vezes que o estado $A$ aparece na história $h$.

\section{Problema do rio}

Nessa seção, definimos o problema abstrato que serviu como base para as experimentações: o problema do rio.

No problema do rio, um agente está em um lado da margem do rio e quer ir para o outro lado. O agente tem duas opções:
(i) nadar de qualquer ponto da margem do rio, ou (ii) ir ao longo da margem do rio até uma ponte. Este problema é modelado
como uma grade $N_x$ $\times$ $N_y$, onde $x = 0$ e $x = N_x-1$ representa a margem do rio; e $y = N_y-1$ representa a ponte e $y = 0$ representa uma cachoeira onde o agente pode ficar preso ou morrer. O objetivo fica do outro lado da margem do rio, longe do
ponte, $x_g = N_x-1$ e $y_g = 0$. O agente pode se mover na direções cardinais. Se ações são tomadas na margem do rio ou
na ponte, então as transições são determinísticas em relação às respectivas direções cardeais; se ações são tomadas no rio então
as transições são probabilísticas e seguem a direção cardeal com probabilidade $1 - P_{river}$ ou segue rio abaixo com probabilidade $P_{river}$ . A cachoeira é modelada como um estado \textit{dead end}. O custo imediato é $1$. Veja a imagem abaixo para a visualização do problema.

\begin{figure}[h]
  \begin{center}
    \includegraphics[width=0.95\textwidth]{img/1.png}
  \end{center}
  \caption{}
  \label{fig:}
\end{figure}


\section{Intepretabilidade de políticas}
A primeira classe de problema é a classe de interpretabilidade de políticas. Essa classe envolve problemas nos quais o agente fornece informações para o usuário a fim de resumir o funcionamento de uma política. Essa seção está estruturada da forma pergunta + resposta, onde cada pergunta representa um problema que estamos interessados para o agente. Uma solução é dada para cada pergunta tendo em vista o problema do rio.

Agora, descrevemos elementos sobre o processo de explicação do agente. Para isso, devemos definir uma linguagem na qual agente e usuário podem se comunicar. 

Definimos uma \textbf{semântica de evento} que servirá como a interface de consultas de explicações entre agente e usuário. A ideia é que a semântica de evento defina um formato de eventos que temos interesse no processo de interpretabilidade de políticas.  

Um usuário pode não ter interesse nos detalhes internos de uma política dada por um agente. Mas tem interesse em um evento de mais alto nível. Por exemplo, considere um agente inteligente que fornece uma rota da casa de Bob até o aeroporto. O agente fornece uma política que descreve como Bob deve agir em cada rua e curva. O agente pode não estar particularmente interessado, por exemplo, se ele deve virar à direita ou à esquerda na avenida principal. De certa forma, esse aspecto da política não gera "dúvida" para o usuário e essa explicação não é necessária. Mas, ele poderia estar interessado se vai passar por um bairro $x$ ou não. Então, o agente pode dar essa resposta ao usuário, dando a ele uma compreensão maior do funcionamento da política. Esse é um exemplo de um problema de \textbf{interpretabilidade de políticas}. Nessa classe de problemas, o agente fornece informações pontuais para o usuário como forma de resumir a política. 

Dado o resultado do agente, uma política, juntamente com o MDP correspondente, obtém-se uma cadeia de Markov que descreve completamente o processo. Essa cadeia gera uma distribuição sobre histórias, que por sua vez gera uma distribuição sobre eventos de interesse. Fornecer a distribuição completa sobre histórias não é razoável na maioria dos casos pois exige um esforço cognitivo indefinidamente grande do usuário. Portanto, buscamos formas alternativas de resumir a política para o usuário.

Voltando para a semântica de evento. Definimos o conjunto $E$ como o conjunto de eventos de interesse no processo. A partir desses eventos, podemos descrever problemas de interpretabilidade e argumentação de políticas.


\subsection{Problema 1}
A primeira pergunta é uma simples. Dada uma semântica bem simples para o conjunto de eventos, queremos saber se um evento acontece. Mais especificamente:

\begin{center}
\boxed{
\begin{minipage}{\textwidth}

  \textbf{Problema 1}. Seja a semântica de evento definida por: $e(A)$ é o evento "o processo passou pelo estado $A$ em algum momento". Mais formalmente,
  \[e(A) = \{h : \exists t : s(h,t) = A\}\]

  Sejam $M = (S, A, T, G, C)$ um MDP e $A$ um estado $\in S$. Determine a probabilidade do evento $e(A)$. 

\end{minipage}
}
\end{center}

\textbf{Solução.} Introduz estado absorvedor artificíal em A e calcula as probabilidades limites.

\subsection{Problema 2}

\begin{center}
\boxed{
\begin{minipage}{\textwidth}

  \textbf{Problema 2}. Seja a semântica de evento definida por: $e(A)$ é o evento "o processo passou pelo estado $A$ em algum momento". Mais formalmente,
  \[e(A) = \{h : \exists t : s(h,t) = A\}\]

  Sejam $M = (S, A, T, G, C)$ um MDP, $A$ um conjunto de eventos \in $S$, $A = \{E_0,E_1,...,E_k\}$ e $\phi(A)$ uma fórmula lógica na forma normal conjuntiva sobre esse conjunto $\phi(A)$. Determine a probabilidade de $\phi(A)$. 

\end{minipage}
}
\end{center}

A solução desse problema permite consideramos fórmulas condicionais sobre eventos. Por exemplo, se temos interesse em saber se um evento implica outro, instânciamos o \textbf{problema $2$} com $\phi(\{A,B\}) = \neg A \lor B $, que é equivalente a $A \rightarrow B$.

\subsection{Problema 3}

\begin{center}
\boxed{
\begin{minipage}{\textwidth}

  \textbf{Problema 3}. Seja a semântica de evento definida por: $e(A,n)$ é o evento "o processo passou pelo estado $A$ $n$ vezes". Mais formalmente,
  \[e(A) = \{h : \exists t_i : s(h,t_i) = A, i \in \{1,2,\dots,n\}\}\]

  Sejam $M = (S, A, T, G, C)$ um MDP, $A$ um estado $\in S$ e $n \in \mathbb{Z}_{+}$. Determine a probabilidade do evento $e(A,n)$.

\end{minipage}
}
\end{center}

Com esse valor, podemos calcular quantas vezes, em média, o processo passa pelo estado $A$:

\[\mathbb{E}[f(h, A)] = \sum_{i = 0}^{\infty} ie(A,i)\]

Além disso, uma quantidade de interesse é o número de vezes, em média, que o processo passa pelo estado $A$, dado que passou por ele alguma vez. Podemos calcular essa quantidade a partir dos valores de $e(A,i)$:
\begin{align}
  \mathbb{E}[f(h,A) | f(h,A) \ge 1]=& \sum_{i = 0}^{\infty} i \Pr\{f(h,A) = i | f(h,A) \ge 1\} \\
=& \sum_{i = 0}^{\infty} i \frac{\Pr\{f(h,A) = i , f(h,A) \ge 1\}}{\Pr\{f(h,A) \ge 1\}} \\
=& \frac{\sum_{i = 1}^{\infty} i \Pr\{e(A,i)\}}{\sum_{i = 1}^{\infty} \Pr\{e(A,i)\}}
\end{align}

\section{Argumentação}

Outro aspecto importante do processo de explicação do agente é o de argumentação. Nesse cenário, o agente é questionado sobre alguma decisão. No exemplo do Bob, o usuário poderia perguntar para o agente porque ele escolheu uma política que passa pelo bairro $x$ e não pelo bairro $y$. A resposta do agente depende do critério de decisão do processo. Por exemplo, se o objetivo do usuário é uma rota mais curta possível, o agente pode explicar para o usuário que qualquer rota passando por $y$ é mais longa que a rota fornecida (como forma de política).

Para cada pergunta, a resposta pode estar dentro de um de três elementos do MDP: $G$, $T$ ou $C$. Por exemplo, o usuário pode perguntar: Por que o evento A acontece? Analisando a partir da função custo, o agente poderia responder: "Se não A, então o custo esperado aumenta". Olhando para a função objetivo: "Se não A, a probabilidade de atingir um objetivo é pequena". E assim por diante. A partir dessas ideias, podemos formular perguntas usuário-agente.

\subsection{Problema 4}

\begin{center}
\boxed{
\begin{minipage}{\textwidth}

  \textbf{Problema 4}. Seja a semântica de evento definida por ?: 

  Sejam $M = (S, A, T, G, C)$ um MDP e $A$ um evento. Compare o custo de históricos "com não $A$" versus históricos "com $A$". 
\end{minipage}
}
\end{center}


\subsection{Problema 5}

\begin{center}
\boxed{
\begin{minipage}{\textwidth}

  \textbf{Problema 5}. Seja a semântica de evento definida por ?: 

  Sejam $M = (S, A, T, G, C)$ um MDP, $A$ um conjunto de eventos \in $S$, $A = \{E_0,E_1,...,E_k\}$ e $\phi(A)$ uma fórmula lógica na forma normal conjuntiva sobre esse conjunto $\phi(A)$. Compare o custo de históricos "com não $\phi(A)$" versus históricos "com $\phi(A)$".

\end{minipage}
}
\end{center}


\end{document}
